%---------------------------------------------------------------------------------------------------
% Einf�hrung
%---------------------------------------------------------------------------------------------------
\newpage
%\part{Anfang}
\chapter{Einf�hrung}
%Hier muss was zur \addIndexEntry{Einf�hrung} erz�hlt werden.
%In dieser Bachelorthesis geht es um die Machbarkeit einer Echtzeit-Zeichenerkennung auf einem Einplatinencomputer. Im speziellen wird die Tesseract OCR (Optical Character Recognition, Optische Zeichenerkennung) Bibliothek genutzt. 

In meiner Bachelorthesis geht es um die Entwicklung eines Bildschirmleseger�tes auf Basis eine aktuellen Einplatinencomputers. Ein Bildschirmleseger�t ist eine Kombination von Kamerasystem und Monitor. Es ist eine Lesehilfe f�r Menschen mit eingeschr�nktem Sehverm�gen. W�hrend auf dem Markt befindliche Ger�te meist �ber eine Kamera mit gro�em optischem Zoom verf�gen, m�chte ich hier einen anderen Ansatz  verfolgen. Heutige Einplatinencomputer in der Gr��enordnung einer EC-Karte verf�gen bereits �ber Mehrkernprozessoren und somit eine beachtliche Rechenleistung. Daher habe ich im Rahmen dieser Arbeit untersucht in wieweit sich mit einem Foto eines Textes eine Zeichenerkennung durchf�hren l�sst. Eine Zeichenerkennung bietet einige Vorteilen, so kann der erkannte Text beliebig skaliert und auch unabh�ngig vom Layout des urspr�nglichen Textes angezeigt werden. 

\bigskip


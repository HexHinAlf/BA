%---------------------------------------------------------------------------------------------------
% Analyse
%---------------------------------------------------------------------------------------------------
\newpage
\chapter{Analyse}
In diesem Kapitel werden verschiedene Bibliotheken auf ihre Nutzbarkeit f�r mein Vorhaben gepr�ft und bewertet.


\section{Hardware}
\subsection{Raspberry Pi}
Der Raspberry Pi ist ein Einplatinencomputer in der Gr��e einer Kreditkarte. Entwickelt wurde er von der britischen Stiftung Raspberry Pi Fundation um inbesondere jungen Leuten eine kosteng�nstige Platform zum aneignen von Programmierkenntnissen zu bieten. F�r meine Bachelorthesis habe ich den Raspberry Pi 2 Model B  genutzt. Der Verkaufspreis f�r dieses Modell betr�gt in Deutschland etwa 38~EUR (11/2015), ohne zum Betrieb n�tiges Zubeh�r wie Netzteil oder microSD-Karte.

%Tabelle
\begin{table}[h!]
	\begin{center}
		\begin{tabularx}{0.7\textwidth}{| p{1.5cm} | X |}
			\hline
			Gr��e & 93,0~mm $\times$ 63,5~mm $\times$ 20,0~mm (L$\times$B$\times$H)\\
			\hline
			SOC & Broadcom BCM2836 \\
			\hline
			CPU & ARM Cortex-A7, 4$\times$ 900MHz\\
			\hline
			GPU & Broadcom VideoCore IV\\
			\hline
			RAM & 1GB LPDDR2-SDRAM\\
			\hline
			LAN & Microchip LAN9514, 10/100 MBit\\
			\hline
			IO & HDMI 1.4, 4$\times$ USB2.0\\
			\hline
		\end{tabularx}
	\end{center}
	\caption{Spezifikation Raspberry Pi 2 Model B}
	\label{2.1Tabelle}
\end{table}

Als Hauptspeicher kommt microSD-Karte von SanDisk (16~GB, Class 10) zum Einsatz. 



\section{Optische Zeichenerkennung (OCR)}

Bibliotheken, Tests


\subsection{Wie liest ein Mensch?}



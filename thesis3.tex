%---------------------------------------------------------------------------------------------------
% Voreinstellungen (Layout, neue Befehle, etc.)
%---------------------------------------------------------------------------------------------------

\input{einstellungen/grundeinstellungen}															% Die stilistischen Parameter

%--------------------------------------------------------------------------------------------------- 
% Anfang des Schriftst�cks
%---------------------------------------------------------------------------------------------------	
\begin{document}

%--------------------------------------------------------------------------------------------------- 
% Erstellen des Deck- und des Titelblatts
%---------------------------------------------------------------------------------------------------
	\createCoverAndTitlePage{Bachelor}																			% Art der Arbeit
													{thesis}																			% Bezeichnung arbeit oder thesis 
													{Felix Hahn}														% Author
													{Entwicklung und Aufbau eines\\
													 Bildschirmleseger�tes\\
													 auf Basis eines\\
													 Einplatinencomputers}											  % Titel				
													{Informations- und Elektrotechnik}						% Studiengang
													{Prof.\ Dr.\ rer.\ nat.\ Henning Dierks}					% Erstgutachter
													{Prof.\ Dr.\-Ing.\ Annabella Rauscher-Scheibe}								% Zweitgutachter
													

  \createAbstract					{Bachelor}																			% Art der Arbeit
													{thesis}																			% Bezeichnung arbeit oder thesis 
													{Felix Hahn}														% Author
													{Entwicklung und Aufbau eines
													 Bildschirmleseger�tes
													 auf Basis eines
													 Einplatinencomputers}											  % Titel				
													{Development and Construction of a Microprocessor 
													controlled allocation processor}							% Titel Englisch
													{Steuerung, und viele weitere interessante 
													Stichwort}																		% Stichworte
													{Controller, Microprocessor, and other 
													interesting words describing the whole 
													process}																			% Keywords (Stichworte Emglisch)
													{Diese Arbeit umfasst alles was man mit einem 
													Mikrorechner machen kann und nat�rlich noch vieles mehr.
													etc.}																					% Kurzzusammenfassung
													{Inside this report the construction of a very 
													important Controller for microprocessors is 
													described.												
													etc.}																			% Abstract (Kurzzusammenfassung Englisch)

												
%--------------------------------------------------------------------------------------------------- 
% Zusammenfassung
%---------------------------------------------------------------------------------------------------			
  									  													


%--------------------------------------------------------------------------------------------------- 
% Danksagung  
%---------------------------------------------------------------------------------------------------	
	\input{standard/danke}  																							

%--------------------------------------------------------------------------------------------------- 
% Verzeichnisse
%---------------------------------------------------------------------------------------------------	
  \tableofcontents                              												% Inhaltsverzeichnis
	\listoftables                                 												% Tabellenverzeichnis
	\listoffigures                                												% Abbildungsverzeichnis  
	
%--------------------------------------------------------------------------------------------------- 
% Der erste Teil der Arbeit:
%---------------------------------------------------------------------------------------------------
	%---------------------------------------------------------------------------------------------------
% Einf�hrung
%---------------------------------------------------------------------------------------------------
\newpage
%\part{Anfang}
\chapter{Einf�hrung}
%Hier muss was zur \addIndexEntry{Einf�hrung} erz�hlt werden.
%In dieser Bachelorthesis geht es um die Machbarkeit einer Echtzeit-Zeichenerkennung auf einem Einplatinencomputer. Im speziellen wird die Tesseract OCR (Optical Character Recognition, Optische Zeichenerkennung) Bibliothek genutzt. 

In meiner Bachelorthesis geht es um die Entwicklung eines Bildschirmleseger�tes auf Basis eine aktuellen Einplatinencomputers. Ein Bildschirmleseger�t ist eine Kombination von Kamerasystem und Monitor. Es ist eine Lesehilfe f�r Menschen mit eingeschr�nktem Sehverm�gen. W�hrend auf dem Markt befindliche Ger�te meist �ber eine Kamera mit gro�em optischem Zoom verf�gen, m�chte ich hier einen anderen Ansatz  verfolgen. Heutige Einplatinencomputer in der Gr��enordnung einer EC-Karte verf�gen bereits �ber Mehrkernprozessoren und somit eine beachtliche Rechenleistung. Daher habe ich im Rahmen dieser Arbeit untersucht in wieweit sich mit einem Foto eines Textes eine Zeichenerkennung durchf�hren l�sst. Eine Zeichenerkennung bietet einige Vorteilen, so kann der erkannte Text beliebig skaliert und auch unabh�ngig vom Layout des urspr�nglichen Textes angezeigt werden. 

\bigskip


	
%---------------------------------------------------------------------------------------------------	
% Der zweite Teil der Arbeit:
%---------------------------------------------------------------------------------------------------
	\input{kapitel/hauptteil/hauptteil.tex}

%---------------------------------------------------------------------------------------------------
% Der dritte Teil der Arbeit
%---------------------------------------------------------------------------------------------------
%	\input{kapitel/schluss/schluss.tex}
																								
%---------------------------------------------------------------------------------------------------	
% Literaturverzeichnis
%---------------------------------------------------------------------------------------------------		
	\bibliographystyle{dinat}        		    														% Anpassung an deutsche Zitierweise
                                          														% Alphabetische Sortierung, Abk�rzungen
  \bibliography{literatur/literatur}      														% Literaturverzeichnis
  \input{literatur/literatur}																				% hier k�nnen alle Schriftst�cke aufgef�hrt werden, die nicht zitiert, aber dennoch nennenswert sind!

%---------------------------------------------------------------------------------------------------	
% Anh�nge
%---------------------------------------------------------------------------------------------------	
	\appendix
% \input{anhang/hilfsmittel/hilfsmittel}															% Anhang A: Hilfsmittel zur Erstellung
																																			% 					dieser Arbeit
%	\input{anhang/quellcode/quellcode}																	% Anhang B: Quellcode

%---------------------------------------------------------------------------------------------------	
% Glossar
%---------------------------------------------------------------------------------------------------	
	\printnomenclature

%---------------------------------------------------------------------------------------------------	
% Stichwortverzeichnis
%---------------------------------------------------------------------------------------------------	
	\printindex
	
%---------------------------------------------------------------------------------------------------	
% Erkl�rung �ber Selbstst�ndigkeit
%---------------------------------------------------------------------------------------------------		
	\asurency	

%--------------------------------------------------------------------------------------------------- 
% Ende des Schriftst�cks
%--------------------------------------------------------------------------------------------------- 
\end{document}
